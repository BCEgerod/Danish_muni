\begin{table}[h]
	\centering
	\caption{Indicators of Fiscal Policy Conservatism}
	\label{tab:policies}
	\begin{tabular}{lcc} \hline
		\textbf{Policy}                          & \textbf{Availability} & \textbf{Direction} \\
		\hline
		&&\\ \textit{Tax policy} &&\\
		Income tax (pct.)                        & 1974 -     &    Lower       \\
		Property tax (per mille)                      & 1974 -     &    Lower        \\
		Tax on the cost of commercial real estate (per mille) & 1993 -     &    Lower               \\ \hline
	
		&&\\ \textit{Spending policy}  &&\\
		Spending pr. capita (DKK)                & 1974 -     &    Lower        \\
		Spending pr. pupil in school (DKK)       & 1993 -     &    Lower     \\ \hline
		
		&&\\\textit{Organization of public service delivery}  &&\\
 		Public Employees (pr. 1,000 citizens)	 & 1993 -  	  &	   Lower	     \\
 		Privately operated municipal services  (pct.) & 1993 -     &    Higher     \\
 		Purchases with a private supplier  (pct.)                   & 1993 -     &    Higher     \\ \hline
 		&&\\ \textit{Co-payment for public services} &&\\   
		Average cost of day care (DKK)                  & 1993 -     &    Higher     \\
		Price of relief stay (DKK)				 & 1993 -	  &	   Higher	 \\
		Food delivery for the  elderly (DKK) & 1993 -     &    Higher     \\
		Stay in nursing home (DKK)              & 1993 -     &    Higher     \\ \hline
	&&\\ \textit{Extent of Public Services} &&\\ 
		Public housing (pct.)                    & 1993 -     &    Lower               \\
		Class size in public schools	         & 1993 -     &    Lower       \\
		\hline \hline
		\multicolumn{3}{p{14 cm}}{\emph{Note: The table outlines the variables used to capture fiscal policy conservatism in Danish municipalities and their period of availability. All variables are rescaled to have mean zero and variance one, and -- when appropriate -- reversed to have the same direction.}}
	\end{tabular}
\end{table} 